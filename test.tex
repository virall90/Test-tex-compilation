% mathenv.tex - an example showing the different math environments, how
%               to use them and how they look.
%
% Andrew Roberts - 27th April 2004

\documentclass[a4paper]{article}

\usepackage{times}

% The mathptmx package does for maths equations what the times package
% does for the main text. That is, uses scalable fonts rather than the
% default bitmapped ones. This is useful if you later want to convert
% your postscript file to PDF.

\usepackage{mathptmx}

\begin{document}
\title{Using Maths Environments}
\author{Andrew Roberts}
\date{}
\maketitle

Maths is a pretty fundamental area with Latex, and with Tex! Normally,
environments require the standard \verb|\begin{...} \end{...}| format.
However, as it was assumed that maths stuff would be frequent in most
documents you created, then some short cuts were also added. You are
free to use the standard approach, but there are also two shortcuts: one
being the Latex way, and the other being the Tex way. 
